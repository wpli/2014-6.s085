%
% This is the LaTeX template file for lecture notes for CS294-8,
% Computational Biology for Computer Scientists.  When preparing 
% LaTeX notes for this class, please use this template.
%
% To familiarize yourself with this template, the body contains
% some examples of its use.  Look them over.  Then you can
% run LaTeX on this file.  After you have LaTeXed this file then
% you can look over the result either by printing it out with
% dvips or using xdvi.
%

\documentclass[twoside]{article}
\usepackage{graphics}
\usepackage{url}
\usepackage{amsmath}
\setlength{\oddsidemargin}{0.25 in}
\setlength{\evensidemargin}{-0.25 in}
\setlength{\topmargin}{-0.6 in}
\setlength{\textwidth}{6.5 in}
\setlength{\textheight}{8.5 in}
\setlength{\headsep}{0.75 in}
\setlength{\parindent}{0 in}
\setlength{\parskip}{0.1 in}

%
% The following commands set up the lecnum (lecture number)
% counter and make various numbering schemes work relative
% to the lecture number.
%
\newcounter{lecnum}
\renewcommand{\thepage}{\thelecnum-\arabic{page}}
\renewcommand{\thesection}{\thelecnum.\arabic{section}}
\renewcommand{\theequation}{\thelecnum.\arabic{equation}}
\renewcommand{\thefigure}{\thelecnum.\arabic{figure}}
\renewcommand{\thetable}{\thelecnum.\arabic{table}}

%
% The following macro is used to generate the header.
%
\newcommand{\lecture}[4]{
   \pagestyle{myheadings}
   \thispagestyle{plain}
   \newpage
   \setcounter{lecnum}{#1}
   \setcounter{page}{1}
   \noindent
   \begin{center}
   \framebox{
      \vbox{\vspace{2mm}
    \hbox to 6.28in { {\bf 6.S085 Statistics for Research Projects
                        \hfill IAP 2014} }
       \vspace{4mm}
       \hbox to 6.28in { {\Large \hfill Lecture #1: #2  \hfill} }
       \vspace{2mm}
       \hbox to 6.28in { {\it Lecturer: #3 \hfill Notes by: #4} }
      \vspace{2mm}}
   }
   \end{center}
   \markboth{Lecture #1: #2}{Lecture #1: #2}
   {\bf Disclaimer}: {\it These notes have not been subjected to the
   usual scrutiny reserved for formal publications.  They may be distributed
   outside this class only with the permission of the Instructor.}
   \vspace*{4mm}
}

%
% Convention for citations is authors' initials followed by the year.
% For example, to cite a paper by Leighton and Maggs you would type
% \cite{LM89}, and to cite a paper by Strassen you would type \cite{S69}.
% (To avoid bibliography problems, for now we redefine the \cite command.)
% Also commands that create a suitable format for the reference list.
\renewcommand{\cite}[1]{[#1]}
\def\beginrefs{\begin{list}%
        {[\arabic{equation}]}{\usecounter{equation}
         \setlength{\leftmargin}{2.0truecm}\setlength{\labelsep}{0.4truecm}%
         \setlength{\labelwidth}{1.6truecm}}}
\def\endrefs{\end{list}}
\def\bibentry#1{\item[\hbox{[#1]}]}

%Use this command for a figure; it puts a figure in wherever you want it.
%usage: \fig{NUMBER}{SPACE-IN-INCHES}{CAPTION}
\newcommand{\fig}[3]{
			\vspace{#2}
			\begin{center}
			Figure \thelecnum.#1:~#3
			\end{center}
	}
% Use these for theorems, lemmas, proofs, etc.
\newtheorem{theorem}{Theorem}[lecnum]
\newtheorem{lemma}[theorem]{Lemma}
\newtheorem{proposition}[theorem]{Proposition}
\newtheorem{claim}[theorem]{Claim}
\newtheorem{corollary}[theorem]{Corollary}
\newtheorem{definition}[theorem]{Definition}
\newenvironment{proof}{{\bf Proof:}}{\hfill\rule{2mm}{2mm}}

% **** IF YOU WANT TO DEFINE ADDITIONAL MACROS FOR YOURSELF, PUT THEM HERE:

\begin{document}
%FILL IN THE RIGHT INFO.
%\lecture{**LECTURE-NUMBER**}{**DATE**}{**LECTURER**}{**SCRIBE**}
\lecture{7}{January 29}{Ramesh Sridharan and George Chen}{William Li}

% **** YOUR NOTES GO HERE:

% Some general latex examples and examples making use of the
% macros follow.  
%**** IN GENERAL, BE BRIEF. LONG SCRIBE NOTES, NO MATTER HOW WELL WRITTEN,
%**** ARE NEVER READ BY ANYBODY.
\section{Experimental Design}

Simple random sampling (SRS): randomly choose subset of individuals from a population (N people) without replacement

Sample $n$ individuals: any subset of $n$ individuals is equally likely

Blocking/controlling for confounding variables: dealing with Simpson's paradox

Randomization: need this often to get independent samples and to meet assumptions for stat. tests

Replication: good experiments should be reproducible! Replacing experiment should yield similar reuslts

Control/baseline comparison: to measure effect of treatment, need ref to compare against.

\section{Stratified Random Sampling}

Block things out, and do simple random sampling (SRS) in each block

For a stratum: how large of a simple random sampling?

\begin{definition}[Proportional Allocation]

Number of individuals in a stratum matches stratum's relative size in population

\end{definition}

\begin{definition}[Neyman Allocation]

Combines proportional allocation with looking at variances within a strata

Higher variances $\rightarrow$ larger SRS (higher relative size of stratum in population)

\end{definition}

\begin{definition}[Cluster sampling] 

Previous methods require sampling from whole population or every stratum

Idea: partition population into ``natural'' groups (each group well-represents the population)

\end{definition}

Cluster Sampling:

\begin{enumerate}
\item Randomly sample a few groups
\item For each chosen group, obtain SRS
\end{enumerate}

Example: Polling a city

\begin{enumerate}
\item Divide city into blocks
\item Randomly choose a few blocks
\item Within each chosen block, get SRS
\end{enumerate}

\section{Problem Set Discussion}

$\log{y} = \beta_1 \log{x} + \beta_0 + \epsilon$

$y \propto \exp{beta_0} x^{\beta_1}$

Bat example:

Higher residuals for one group as opposed to another: the model may not be taking into account some factors

\section{Issues (with all Survey Methods)}

\begin{itemize}
\item Getting a list of everyone in population is often hard
\item Non-response bias: many do not respond
\item Question wording matters
\end{itemize}

\subsection{Alfred Landon and Literary Digest}

57\% of respondents supported Landon over Roosevelt in 1936 (biased sample) 

\section{Placebo Effect}

Person gets a sham/dud treatment that is advertised as effective

Then the person reports feeling better (therapeutic effect)

\section{Paired Test, Repeated Measures}

e.g. before, after

2 or more treatments that each subject receives

Example: Want ot measure the effect of caffeine

1 month no caffeine

1 month coffee

1 month tea

Note: there are temporal effects (fancier mehods for adding wit hthis -- autocorrelation models)

%\begin{definition}              
%
%Hello World                     
%
%\end{definition}

% This is a comment
% that I want to continue
% OK great


\end{document}






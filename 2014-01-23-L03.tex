%
% This is the LaTeX template file for lecture notes for CS294-8,
% Computational Biology for Computer Scientists.  When preparing 
% LaTeX notes for this class, please use this template.
%
% To familiarize yourself with this template, the body contains
% some examples of its use.  Look them over.  Then you can
% run LaTeX on this file.  After you have LaTeXed this file then
% you can look over the result either by printing it out with
% dvips or using xdvi.
%

\documentclass[twoside]{article}
\usepackage{graphics}
\usepackage{url}
\usepackage{amsmath}
\setlength{\oddsidemargin}{0.25 in}
\setlength{\evensidemargin}{-0.25 in}
\setlength{\topmargin}{-0.6 in}
\setlength{\textwidth}{6.5 in}
\setlength{\textheight}{8.5 in}
\setlength{\headsep}{0.75 in}
\setlength{\parindent}{0 in}
\setlength{\parskip}{0.1 in}

%
% The following commands set up the lecnum (lecture number)
% counter and make various numbering schemes work relative
% to the lecture number.
%
\newcounter{lecnum}
\renewcommand{\thepage}{\thelecnum-\arabic{page}}
\renewcommand{\thesection}{\thelecnum.\arabic{section}}
\renewcommand{\theequation}{\thelecnum.\arabic{equation}}
\renewcommand{\thefigure}{\thelecnum.\arabic{figure}}
\renewcommand{\thetable}{\thelecnum.\arabic{table}}

%
% The following macro is used to generate the header.
%
\newcommand{\lecture}[4]{
   \pagestyle{myheadings}
   \thispagestyle{plain}
   \newpage
   \setcounter{lecnum}{#1}
   \setcounter{page}{1}
   \noindent
   \begin{center}
   \framebox{
      \vbox{\vspace{2mm}
    \hbox to 6.28in { {\bf 6.S085 Statistics for Research Projects
                        \hfill IAP 2014} }
       \vspace{4mm}
       \hbox to 6.28in { {\Large \hfill Lecture #1: #2  \hfill} }
       \vspace{2mm}
       \hbox to 6.28in { {\it Lecturer: #3 \hfill Notes by: #4} }
      \vspace{2mm}}
   }
   \end{center}
   \markboth{Lecture #1: #2}{Lecture #1: #2}
   {\bf Disclaimer}: {\it These notes have not been subjected to the
   usual scrutiny reserved for formal publications.  They may be distributed
   outside this class only with the permission of the Instructor.}
   \vspace*{4mm}
}

%
% Convention for citations is authors' initials followed by the year.
% For example, to cite a paper by Leighton and Maggs you would type
% \cite{LM89}, and to cite a paper by Strassen you would type \cite{S69}.
% (To avoid bibliography problems, for now we redefine the \cite command.)
% Also commands that create a suitable format for the reference list.
\renewcommand{\cite}[1]{[#1]}
\def\beginrefs{\begin{list}%
        {[\arabic{equation}]}{\usecounter{equation}
         \setlength{\leftmargin}{2.0truecm}\setlength{\labelsep}{0.4truecm}%
         \setlength{\labelwidth}{1.6truecm}}}
\def\endrefs{\end{list}}
\def\bibentry#1{\item[\hbox{[#1]}]}

%Use this command for a figure; it puts a figure in wherever you want it.
%usage: \fig{NUMBER}{SPACE-IN-INCHES}{CAPTION}
\newcommand{\fig}[3]{
			\vspace{#2}
			\begin{center}
			Figure \thelecnum.#1:~#3
			\end{center}
	}
% Use these for theorems, lemmas, proofs, etc.
\newtheorem{theorem}{Theorem}[lecnum]
\newtheorem{lemma}[theorem]{Lemma}
\newtheorem{proposition}[theorem]{Proposition}
\newtheorem{claim}[theorem]{Claim}
\newtheorem{corollary}[theorem]{Corollary}
\newtheorem{definition}[theorem]{Definition}
\newenvironment{proof}{{\bf Proof:}}{\hfill\rule{2mm}{2mm}}

% **** IF YOU WANT TO DEFINE ADDITIONAL MACROS FOR YOURSELF, PUT THEM HERE:

\begin{document}
%FILL IN THE RIGHT INFO.
%\lecture{**LECTURE-NUMBER**}{**DATE**}{**LECTURER**}{**SCRIBE**}
\lecture{3}{January 23}{Ramesh Sridharan and George Chen}{William Li}
%\footnotetext{These notes are partially based on those of Nigel Mansell.}

% **** YOUR NOTES GO HERE:

% Some general latex examples and examples making use of the
% macros follow.  
%**** IN GENERAL, BE BRIEF. LONG SCRIBE NOTES, NO MATTER HOW WELL WRITTEN,
%**** ARE NEVER READ BY ANYBODY.
\section{Linear Regression}

Be sure to visualize! Recall Anscombe's Quartet.

Goal of linear regression: fit a line, $y = \beta_0 + \beta_1 x$ to data

$y$: dependent variable/response variable

$\beta_0$: intercept

$\beta_1$: slope (e.g. spring constant in Hooke's law)

$x$: independent variable/predictor variable

Slope close to 0: little/no relationship between $x$ and $y$

\section{Probabilistic Model}

Observe $(x_1,y_1), (x_2,y_2), ... , (x_n,y_n)$

Assume:

$y_i(x_i) = \beta_0 + \beta_1 x_i + \epsilon_i$

$\epsilon_i$: normally distributed with mean 0, variance $\sigma^2$ ($N(0,\sigma^2)$)

$\beta_0$ and $\beta_1$ are fixed but unknown.

Under this model, there is a ``good'' way of to estimate $\beta_0$, $\beta_1$:

$\min\limits_{\beta_0,\beta_1} \sum\limits_{i=1}^{n} ( y - (\beta_0 + \beta_1 x_i ) )^2$

(This is called least-squares linear regression)

Solution: 

$\hat{\beta}_1 = \frac{\sum\limits_{i=1}^n x_i y_i - \frac{1}{n}(\sum\limits_{i=1}^n x_i)(\sum\limits_{i=1}^n y_i)}{\sum\limits_{i=1}^n x_i^2 - \frac{1}{n}(\sum\limits_{i=1}^n x_i)^2}$

Correlation coefficient: $r = \frac{1}{n-1}\sum\limits_{i=1}^{n}\frac{(x_i - \bar{x})(y_i-\bar{y})}{s_x s_y}$

where $s_x = \sqrt{\frac{1}{n-1} \sum\limits_{i=1}^n(x_i-\bar{x})^2}$

$r^2$ is called the ``coefficient of determination'', $-1 \leq r \leq 1$

Correlation does not imply casuation, e.g. sun cycles was correlated with the number of Republicans in the Senate in the 1980s

\section{Hypothesis Testing}

Warning: Everything here assumes that the above probabilsitic model is true

\subsection{Slope}
$t_{\beta_1} = \frac{\hat{\beta_1} - \beta_1}{s_{\beta_1}} \sim t_{n-2}$ (t distributed with $n-1$ degrees of freedom

$s_{beta_1} = \frac{\hat{\sigma}}{\sqrt{\sum\limits_{i=1}^n (x_i - \bar{x})^2}}$

$\hat{\sigma}^2 = \sum\limits_{i=1}^n \frac{(\hat{y_i} - y_i)^2)^2}{n-2}$

Let's break down these terms:

${\sqrt{\sum\limits_{i=1}^n (x_i - \bar{x})^2}}$: How close together the $x_i$'s are

Intuition:

When the $x_i$ values are very close together, it's hard to fit a line. This makes $s_{\beta_1}$ large. This makes $t_{\beta_1}$ smaller (closer to 0). For a null hypothesis ($H_0$) of $\beta_1 = 0$, there will be more area outside of $[-t_{\beta_1},t_{\beta_1}]$, which means that your statistical significance will go down. (Your p-value will be higher, or it will be harder to achieve your threshold of statistical significance).

When there is a large error in fit, the term $\hat{\sigma}^2$ will be large. This makes $s_{\beta_1} large$... your statistical significance will go down (following the line of reasoning above).

\subsection{Intercept}

$t_{\beta_0} = \frac{\hat{\beta_0} - \beta_0}{s_{\beta_0}} \sim t_{n-2}$

$s_{\beta_0} = \hat{\sigma} \sqrt{\frac{1}{n} + \frac{\bar{x}^2}{\sum\limits_{i=1}^n (x_i - \bar{x})^2}}$

If the $x_i$'s are close together, then the $s_{\beta_0}$ will be big. Then $t_{beta_0}$ will be small. For a null hypothesis of $\beta_0 = 0$, there will be more area outside of $[-t_{\beta_0},t_{\beta_0}]$, which means that your statistical significance will go down (your p-value will be higher).

Bonferroni correction: divide the p-value threshold by 2 to see if you have significance at p=0.05 (i.e. is p<0.025 for both of them?)

\subsection{Correlation Coefficient}

$t_r = \sqrt{\frac{n-2}{1-r^2}} \sim t_{n-2}$

\subsection{Prediction}

What can we say about a new point generated from same probabilistic model with x-value $x^*$?

$\hat{y}(x^*) = \beta_0 + \hat{\beta}_1 x^*$

$var[\hat{\beta}_0 + \hat{\beta}_1 x^* + \epsilon] = var[\hat{\beta}_0 + \hat{\beta}_1x^*] + var[\epsilon]$

The first term is estimated with $\hat{\sigma}^2[\frac{1}{n} + \frac{(x^*-\bar{x})^2}{\sum(x_i-\bar{x})^2}$ --- this is the variance in the mean estimate of the line at $x^*$

The second term is estimated with $\hat{\sigma}^2$ --- it is variance introduced by extra noise

\subsection{Multiple linear regression}
$((x_1,x_2,...,x_p),y)...$ --- there are $p$ predictors for each $y$

Probabilistic model:

$y = \beta_1 x_1 + \beta_2 x_2 + ... + \beta_p x_p + \epsilon$

Why not use a very high degree polynomial? It will overfit.

\subsection{Matrix Form}

Solve:

$\min\limits_{\beta} \sum\limits_{i=1}^n (Y_i - X_i\beta))^2$

Solution:

$\hat{\beta} = (X^TX)^{-1}X^TY$

\section{Model Evaluation}

$y_i - \bar{y} = (\hat{y}_i - \bar{y}) + (y_i-\bar{y})$

First term: difference explained by the model
Second term: difference not explained by the model

Residual: $\hat{y}_i - y_i$: Visualizing this is important

With a bit of algebra:

$\sum\limits_{i=1}^n (y_i-\bar{y})^2 = \sum (\hat{y}_i - \bar{y})^2 + \sum(y_i - \hat{y}_i)^2$

SST = SSM + SSE
(sum of squares total) = ( sum of squares model) + sum of squares error

(sum of squares model) / (sum of squares total) is exactly $r^2$!

We want SSM/SSE to be large

MSM = SSM / $p-1$

MSE = SSE / $n-p$

For technical reasons, instead look at: MSM/MSE $\sim$ $F_{p-1,n-p}$

%\section*{References}           
%\beginrefs
%\bibentry{AGM97}{\sc N.~Alon}, {\sc Z.~Galil} and {\sc O.~Margalit},
%On the Exponent of the All Pairs Shortest Path Problem,
%{\it Journal of Computer and System Sciences\/}~{\bf 54} (1997),
%pp.~255--262.

%\bibentry{F76}{\sc M. L. ~Fredman}, New Bounds on the Complexity of the 
%Shortest Path Problem, {\it SIAM Journal on Computing\/}~{\bf 5} (1976), 
%pp.~83-89.
%\endrefs


\end{document}






%
% This is the LaTeX template file for lecture notes for CS294-8,
% Computational Biology for Computer Scientists.  When preparing 
% LaTeX notes for this class, please use this template.
%
% To familiarize yourself with this template, the body contains
% some examples of its use.  Look them over.  Then you can
% run LaTeX on this file.  After you have LaTeXed this file then
% you can look over the result either by printing it out with
% dvips or using xdvi.
%
% This template is based on the template for Prof. Sinclair's CS 270.

\documentclass[twoside]{article}
\usepackage{graphics}
\usepackage{url}
\usepackage{amsmath}
\setlength{\oddsidemargin}{0.25 in}
\setlength{\evensidemargin}{-0.25 in}
\setlength{\topmargin}{-0.6 in}
\setlength{\textwidth}{6.5 in}
\setlength{\textheight}{8.5 in}
\setlength{\headsep}{0.75 in}
\setlength{\parindent}{0 in}
\setlength{\parskip}{0.1 in}

%
% The following commands set up the lecnum (lecture number)
% counter and make various numbering schemes work relative
% to the lecture number.
%
\newcounter{lecnum}
\renewcommand{\thepage}{\thelecnum-\arabic{page}}
\renewcommand{\thesection}{\thelecnum.\arabic{section}}
\renewcommand{\theequation}{\thelecnum.\arabic{equation}}
\renewcommand{\thefigure}{\thelecnum.\arabic{figure}}
\renewcommand{\thetable}{\thelecnum.\arabic{table}}

%
% The following macro is used to generate the header.
%
\newcommand{\lecture}[4]{
   \pagestyle{myheadings}
   \thispagestyle{plain}
   \newpage
   \setcounter{lecnum}{#1}
   \setcounter{page}{1}
   \noindent
   \begin{center}
   \framebox{
      \vbox{\vspace{2mm}
    \hbox to 6.28in { {\bf 6.S085 Statistics for Research Projects
                        \hfill IAP 2014} }
       \vspace{4mm}
       \hbox to 6.28in { {\Large \hfill Lecture #1: #2  \hfill} }
       \vspace{2mm}
       \hbox to 6.28in { {\it Lecturer: #3 \hfill Notes by: #4} }
      \vspace{2mm}}
   }
   \end{center}
   \markboth{Lecture #1: #2}{Lecture #1: #2}
   {\bf Disclaimer}: {\it These notes have not been subjected to the
   usual scrutiny reserved for formal publications.  They may be distributed
   outside this class only with the permission of the Instructor.}
   \vspace*{4mm}
}

%
% Convention for citations is authors' initials followed by the year.
% For example, to cite a paper by Leighton and Maggs you would type
% \cite{LM89}, and to cite a paper by Strassen you would type \cite{S69}.
% (To avoid bibliography problems, for now we redefine the \cite command.)
% Also commands that create a suitable format for the reference list.
\renewcommand{\cite}[1]{[#1]}
\def\beginrefs{\begin{list}%
        {[\arabic{equation}]}{\usecounter{equation}
         \setlength{\leftmargin}{2.0truecm}\setlength{\labelsep}{0.4truecm}%
         \setlength{\labelwidth}{1.6truecm}}}
\def\endrefs{\end{list}}
\def\bibentry#1{\item[\hbox{[#1]}]}

%Use this command for a figure; it puts a figure in wherever you want it.
%usage: \fig{NUMBER}{SPACE-IN-INCHES}{CAPTION}
\newcommand{\fig}[3]{
			\vspace{#2}
			\begin{center}
			Figure \thelecnum.#1:~#3
			\end{center}
	}
% Use these for theorems, lemmas, proofs, etc.
\newtheorem{theorem}{Theorem}[lecnum]
\newtheorem{lemma}[theorem]{Lemma}
\newtheorem{proposition}[theorem]{Proposition}
\newtheorem{claim}[theorem]{Claim}
\newtheorem{corollary}[theorem]{Corollary}
\newtheorem{definition}[theorem]{Definition}
\newenvironment{proof}{{\bf Proof:}}{\hfill\rule{2mm}{2mm}}

% **** IF YOU WANT TO DEFINE ADDITIONAL MACROS FOR YOURSELF, PUT THEM HERE:

\begin{document}
%FILL IN THE RIGHT INFO.
%\lecture{**LECTURE-NUMBER**}{**DATE**}{**LECTURER**}{**SCRIBE**}
\lecture{1}{January 21}{Ramesh Sridharan and George Chen}{William Li}
%\footnotetext{These notes are partially based on those of Nigel Mansell.}

% **** YOUR NOTES GO HERE:

% Some general latex examples and examples making use of the
% macros follow.  
%**** IN GENERAL, BE BRIEF. LONG SCRIBE NOTES, NO MATTER HOW WELL WRITTEN,
%**** ARE NEVER READ BY ANYBODY.
\section{Motivating Examples}

\begin{itemize}
\item polling probability: extreme statistical anomalies with random polling 
\item URL: \url{http://www.dailykos.com/story/2010/06/29/880179/-Research-2000-Problems-in-plain-sight)}
\item iris recognition: probability of match of iris (used to find the subject of famous National Geographic cover)
\end{itemize}

\section{Introduction: Some Concepts in Statistics}

\subsection{Some Definitions}

Probability: have model/''truth'', want ``what kind of data will this give me?''

Statistics: have data, want to find the underlying model/''truth''

Bayesian: hidden model is random

Frequentist: hidden model is fixed but unknown (``there is some fixed value of the model that exists'')

\textbf{This class focuses on classical frequentist methods.}

\subsection{Types of Data}

Categorical: red/blue, yes/no

Ordinal: anything that can be ordered: disagree/neutral/agree, etc.

Continuous: numerical, any values

Discrete: we will ``lump these in'' with one of the other models (categorical, ordinal, continuous)

\subsection{Random Variables}

Working definition: a quantity that takes on random values

Examples: Height of a randomly chosen student; temperature in January

Probability distributions, i.e. for random variable $x$, $P(x)$

Empirical distribution: based on observed data

Example: Suppose we observe $(1,1,3,4,7,8,8)$; then $p(1) = 2/8$; $p(4) = 1/8$; $p(7) = 2/8$

Expectation: the ``average value'' that a random variable takes, $E[x] = \sum\limits_a a \cdot P(a)$

Expectation is linear, therefore:

$E[ax+by] = aE[x] + bE[y]$

Example: 

\begin{verbatim}
x   y   x+y
------------
1   3   4
2   4   6
5   3   8
4   3   7
3   4   7
\end{verbatim}

(all rows are equally likely)

$E[x] = 15/5 = 3$

$E[y] = 17/5 = 3.4$

$E[x+y] = 32/5 = 6.4$

What if we scramble $x$ and $y$ separately?

\begin{verbatim}
x   y   x+y
------------
1   3   4
2   3   5
3   3   6
4   4   8
5   4   9
\end{verbatim}

$E[x]$, $E[y]$, $E[x+y]$ are the same

No matter how independent/dependent $x$ and $y$ are, expectation is always linear

\subsection{Variance}

$var[x] = \sum\limits_a p(a) \cdot (a - E[x])^2$

$var[ax] = a^2 var[x]$

$var[x+y] = var[x] + var[y]$ IF $x,y$ are independent

standard deviation: $\sqrt{var[x]}$

\subsection{Notation}

$\mu_x$: mean of r.v. $x$

$\sigma_x$: standard deviation of r.v. $x$

$\sigma_x^2$: variance of r.v. $x$ 

\section{Exploratory Analysis}

\subsection{Visualization}
When you get some data, you often want to see what's going on by visualizing the data.

Histogram: count frequency of data

\begin{verbatim}
4
3    x
2 x  x     x
1 x  x  x  x  x
__________________
  1  2  3  4  5
\end{verbatim}

Boxplots: 

\begin{verbatim}
          median        end of data that falls in outlier threshold   
|-----[     |      ]--------|   .   .   .
   quartile  quartile (75%)          outliers
-----------------------------------------
             x 
\end{verbatim}

Cumulative Distribution Function: for random variable $x$, $f(a) = P(x \leq a )$

\begin{itemize}
\item The CDF is a monotonically increasing function
\end{itemize}

For a discrete distribution, it might look something like this:

\begin{verbatim}
4/4|         o----
3/4|      o--.
2/4|   o--.
1/4| ._.
0/4 o------------
\end{verbatim}

Scatter Plot: visualizing two random variables

\begin{verbatim}
  | 
  |     x           x    x
y |        x   x   x   x 
  |          x    x
  |   x        x 
  |
  ____________________________
               x
\end{verbatim}

Recommendation: visualize data before jumping into the analysis (you will catch things that you otherwise wouldn't see)

Examples:
\begin{itemize}
\item Is your data multimodal? If so, then using the mean to summarize it is not helpful
\item Skew: long tail to the right (``right skew''); long tail to the left(``left skew'') -- this will pull the mean further in that direction than the median
\end{itemize}

The median can be more robust to high values

\subsection{Quantitative Measures}

Sample mean: $\hat{\mu}_x = \frac{1}{n} \sum\limits_i x_i$

Sample Variance: $\sigma_x^2 = \frac{1}{n-1} \sum\limits_i (x_i - \hat{\mu}_x)^2$ (note the $n-1$ in denominator)

Median: 50\% of the data is below this value

Mode: most common value

Range: largest - smallest

Is the sample mean a good approximation of the true mean?

\begin{itemize}
\item $x_i$ are random
\item They have fixed but unknown mean $\mu_x$
\item We compute $\hat{\mu}_x$
\end{itemize}

Insight: $\hat{\mu}_x$ is also a random variable

\begin{align}
E[\hat{\mu}_x] & = E[\frac{1}{n} \sum x_i] \\
& = \frac{1}{n} \sum\limits_i \mu_x \\
& = \mu_x
\end{align}

Sample variance: why do we have this $n-1$ term?

We underestimate the sample variance because the $(x_i - \hat{\mu}_x)$ term is too small

$E[\hat{\sigma}_x^2] = \sigma_x^2$

Bias: how ``wrong'' a quantity is

\subsection{Anscombe's Quartet}

Consider 4 datasets with $(x,y)$ pairs:
\begin{itemize}
\item same mean in $x$ and $y$
\item same standard deviation in $x$ and $y$
\item same correlation between $x$ and $y$ 
\end{itemize}

same mean in $x$, same mean in $y$, same std dev in $x$ and $y$, same correlation -- but four very different datasets!

Question: how many summary statistics do you need to summarize a dataset?

\subsection{Gaussian/Normal Distribution}

$p(x) = a \cdot e^{(x-\mu)^2}$

A Gaussian distribution is very concentrated around its mean

We only need the mean and variance to characterize it

Probability of being within one standard deviation of the mean $\approx$ 68\%

Two SDs: 95\%

Three SDs: 99\%

\subsection{Bernoulli Distribution}

binary random variable:

$Pr(x=0) = 1-p$

$Pr(x=1) = p$ 

If $x$ is Bernoulli: 

$x ~ Ber(p)$

$E[x] = p$

$var[x] = p(1-p)$

\subsection{Binomial Distribution}

sum of $n$ independent and identically distributed (i.i.d.) Bernoulli random variables

parameters: $n$ (number of Bernoulli r.v.'s) and $p$ (probability of 1 in Bernoulli r.v.)

If $b$ is binomial

$b ~ B$ (``$b$ is distributed as $B$'')

$E[b] = E[\sum x_i] = np$

$var[b] = np(1-p)$

\subsection{Chi-squared Distribution}

Represented as $\chi^2$

If $x_1,...x_n$, $x_i ~ N(0,1)$

$y = \sum x_i^2$, $y ~ \chi^2 (n)$ 

parameters: $n$ (degrees of freedom)

\subsection{Standard Normal}

$x ~ N(\mu, \sigma^2)$

$y = \frac{x-\mu}{\sigma}$

$y = N(0,1)$

\section{Endnote: 2004 Election}

Bush won all 15 poorest states but only won 36\% of the ``poor vote''

Kerry won 9 out of 11 of the richest states but only won 38\% of the ``rich vote''

There is a confounding factor: rate at which

weak dependence on income in Connecticut, a rich state

strong dependence on income in Mississippi, a poor state

Simpson's paradox

%\section*{References}           
%\beginrefs
%\bibentry{AGM97}{\sc N.~Alon}, {\sc Z.~Galil} and {\sc O.~Margalit},
%On the Exponent of the All Pairs Shortest Path Problem,
%{\it Journal of Computer and System Sciences\/}~{\bf 54} (1997),
%pp.~255--262.

%\bibentry{F76}{\sc M. L. ~Fredman}, New Bounds on the Complexity of the 
%Shortest Path Problem, {\it SIAM Journal on Computing\/}~{\bf 5} (1976), 
%pp.~83-89.
%\endrefs


\end{document}






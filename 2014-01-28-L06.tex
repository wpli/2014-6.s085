%
% This is the LaTeX template file for lecture notes for CS294-8,
% Computational Biology for Computer Scientists.  When preparing 
% LaTeX notes for this class, please use this template.
%
% To familiarize yourself with this template, the body contains
% some examples of its use.  Look them over.  Then you can
% run LaTeX on this file.  After you have LaTeXed this file then
% you can look over the result either by printing it out with
% dvips or using xdvi.
%

\documentclass[twoside]{article}
\usepackage{graphics}
\usepackage{url}
\usepackage{amsmath}
\setlength{\oddsidemargin}{0.25 in}
\setlength{\evensidemargin}{-0.25 in}
\setlength{\topmargin}{-0.6 in}
\setlength{\textwidth}{6.5 in}
\setlength{\textheight}{8.5 in}
\setlength{\headsep}{0.75 in}
\setlength{\parindent}{0 in}
\setlength{\parskip}{0.1 in}

%
% The following commands set up the lecnum (lecture number)
% counter and make various numbering schemes work relative
% to the lecture number.
%
\newcounter{lecnum}
\renewcommand{\thepage}{\thelecnum-\arabic{page}}
\renewcommand{\thesection}{\thelecnum.\arabic{section}}
\renewcommand{\theequation}{\thelecnum.\arabic{equation}}
\renewcommand{\thefigure}{\thelecnum.\arabic{figure}}
\renewcommand{\thetable}{\thelecnum.\arabic{table}}

%
% The following macro is used to generate the header.
%
\newcommand{\lecture}[4]{
   \pagestyle{myheadings}
   \thispagestyle{plain}
   \newpage
   \setcounter{lecnum}{#1}
   \setcounter{page}{1}
   \noindent
   \begin{center}
   \framebox{
      \vbox{\vspace{2mm}
    \hbox to 6.28in { {\bf 6.S085 Statistics for Research Projects
                        \hfill IAP 2014} }
       \vspace{4mm}
       \hbox to 6.28in { {\Large \hfill Lecture #1: #2  \hfill} }
       \vspace{2mm}
       \hbox to 6.28in { {\it Lecturer: #3 \hfill Notes by: #4} }
      \vspace{2mm}}
   }
   \end{center}
   \markboth{Lecture #1: #2}{Lecture #1: #2}
   {\bf Disclaimer}: {\it These notes have not been subjected to the
   usual scrutiny reserved for formal publications.  They may be distributed
   outside this class only with the permission of the Instructor.}
   \vspace*{4mm}
}

%
% Convention for citations is authors' initials followed by the year.
% For example, to cite a paper by Leighton and Maggs you would type
% \cite{LM89}, and to cite a paper by Strassen you would type \cite{S69}.
% (To avoid bibliography problems, for now we redefine the \cite command.)
% Also commands that create a suitable format for the reference list.
\renewcommand{\cite}[1]{[#1]}
\def\beginrefs{\begin{list}%
        {[\arabic{equation}]}{\usecounter{equation}
         \setlength{\leftmargin}{2.0truecm}\setlength{\labelsep}{0.4truecm}%
         \setlength{\labelwidth}{1.6truecm}}}
\def\endrefs{\end{list}}
\def\bibentry#1{\item[\hbox{[#1]}]}

%Use this command for a figure; it puts a figure in wherever you want it.
%usage: \fig{NUMBER}{SPACE-IN-INCHES}{CAPTION}
\newcommand{\fig}[3]{
			\vspace{#2}
			\begin{center}
			Figure \thelecnum.#1:~#3
			\end{center}
	}
% Use these for theorems, lemmas, proofs, etc.
\newtheorem{theorem}{Theorem}[lecnum]
\newtheorem{lemma}[theorem]{Lemma}
\newtheorem{proposition}[theorem]{Proposition}
\newtheorem{claim}[theorem]{Claim}
\newtheorem{corollary}[theorem]{Corollary}
\newtheorem{definition}[theorem]{Definition}
\newenvironment{proof}{{\bf Proof:}}{\hfill\rule{2mm}{2mm}}

% **** IF YOU WANT TO DEFINE ADDITIONAL MACROS FOR YOURSELF, PUT THEM HERE:

\begin{document}
%FILL IN THE RIGHT INFO.
%\lecture{**LECTURE-NUMBER**}{**DATE**}{**LECTURER**}{**SCRIBE**}
\lecture{6}{January 28}{Ramesh Sridharan and George Chen}{William Li}
%\footnotetext{These notes are partially based on those of Nigel Mansell.}

% **** YOUR NOTES GO HERE:

% Some general latex examples and examples making use of the
% macros follow.  
%**** IN GENERAL, BE BRIEF. LONG SCRIBE NOTES, NO MATTER HOW WELL WRITTEN,
%**** ARE NEVER READ BY ANYBODY.

\section{Categorical Data}


\begin{verbatim}
              Outcome
                1      2   
Treatment   1   A      B
            2   C      C (table of counts)
\end{verbatim}


Start with inputs and outputs that are categorical

Usually look at two-way table (contingency table)

Risk: of an outcome for a treatment is proportional of data with tha outcome

``risk fo outcome 1 (cancer) for treatment 1 (smoking) is $A/(A+B)$''

Relative risk: $\frac{A/(A+B)}{C/(C+D)}$

Odds ratio: $\frac{A/B}{C/D}$

\section{Simpson's Paradox}

Data from two hospitals on a risky procedure

\begin{verbatim}
              Live   Die   Survival Rate
Hospital   A   80    120      40% 
           B   20     80      20%
\end{verbatim}

Hospital B sees more patients than A
\begin{verbatim}

GOOD PATIENTS

              Live   Die   Survival Rate
Hospital   A   80    100      44% 
           B   10     10      50%


BAD PATIENTS
              Live   Die   Survival Rate
Hospital   A    0     20       0% 
           B   10     70      13%
\end{verbatim}

Simpson's paradox

Caused by confounding variable at play

\section{Testing Significance of Categorical Data}

We gathered data and avoided confounds. 

Question: 

Is there a relationship between input and output?

Are they not independent?

Null hypothesis: treatment and outcome are independent

Test: 

$\chi^2 = \sum\limits_{entries} \frac{(\text{observed} - \text{expected})^2}{observed}$

Note: in above example, ``hospital'' is the treatment and live/die is the outcome

Expected counts for hospitals:

\begin{verbatim}
A: 2/3
B: 1/3
L: 1/3
D: 2/3

Expected:
     L             D
-------------------------
A   300*A*L     300*A*D

B   300*B*L     300*B*D

     L             D
-------------------------
A    67           133

B    33            67
\end{verbatim}

This is how you compute expected counts for independence

Now, we can go back to the $\chi^2$ formula

For hospital example, $\chi^2 = 12$

This test statistic ($\chi^2$) has a $chi^2$ distribution with $(r-1)(c-1)$ degrees of freedom

We can look it up and see that the p-value is 0.0053; we can reject the null hypothesis that the treatments (hospitals) are independent from the outcome

\subsection{Why is it $\chi^2$}

Each entry is binomial

If the entries are large enough and the samples are indepnedent, each entry is approximately normal

AND the sum of squared standard Gaussians is $\chi^2$

Key assumptions: ``entries are large enough'' and ``samples are independent''

\subsection{What if the entries are too small?}

Fisher's exact test: Works for 2x2 tables

Permutation test

Fisher p-value: $p = \frac{\binom{A+B}{A}\binom{C+D}{C}}{\binom{N}{A+C}}$

Easy if entries are small

Monto Carlo approximation

Yates correction: substract .5 (makes approximation more accurate)

Recall: 

$\binom{n}{k} = \frac{n!}{k!(n-k)!}$

\section{Categorical Inputs, Numerical Outputs}

\subsection{Vocabulary}

Factor: categorical varaible (e.g. color)

Level: value that the factor takes (e.g. ``red'')

\subsection{ANOVA}

Start with one factor, $k$ levels

e.g. input data is color: red/blue/yellow

\begin{verbatim}
Data

MM color    taste score
R           8.1
R           8.0
B           2.1
Y           0.0
B           1.9
\end{verbatim}

Recall, in linear regression:

$y = X\beta + \epsilon$

Now, let's have one predictor per level 

$R = ( 1, 0, 0 )$

$B = ( 0, 1, 0 )$

$Y = ( 0, 0, 1 )$

\[ X = \left( 
\begin{array}{ccc}
1 & 0 & 0 \\
1 & 0 & 0 \\
0 & 1 & 0 \\
0 & 0 & 1 \\
0 & 1 & 0  \end{array}
\right) \]

\[ y = \left( 
\begin{array}{c}
8.1 \\
8.0 \\
2.1 \\
0 \\
1.9  \end{array}
\right) \]


fit linear regression with $X, y$ to get $\beta$

\[ \beta = \left( 
\begin{array}{c}
8.05 \\
2.0 \\
0   \end{array}
\right) \]

Idea: use F-test (``how good is the model?'')

$SS_{total} = SS_{model} + SS_{error}$ 

F-test: how well does the model explain $\rightarrow$ ANOVA!

Is there a relationship between the input and the output?

The question you are asking: Do the categories predict the outcomes? (Not the difference in the categories)

Null hypothesis: $\beta_1 = \beta_2 = ... = 0$

\subsection{What assumptions are we making?}

\begin{itemize}
\item errors independent $\rightarrow$ data must be independent
\item errors must be normally distributed $\rightarrow$ data in each group must be normally distributed
\item All errors have the same variance $rightarrow$ all groups (categories) have the same variance (homoskedasticity)
\end{itemize}

$\hat{\epsilon} = y - \hat{y}$ (residual)

$\epsilon$ (error)

\subsection{Two-Way ANOVA}

Two input factors

Additive (no interaction)

\[ X = \left( 
\begin{array}{cccccc}
R & B & Y & sq & tr & st \\
1 & 0 & 0 & 1 & 0 & 0 \\
1 & 0 & 0 & 0 & 0 & 1 \\
0 & 1 & 0 & 0 & 1 & 0 \\
0 & 0 & 1 & 0 & 0 & 1 \\
0 & 1 & 0 & 0 & 0 & 1 \end{array}
\right) \]

One predictor per shape plus one per color

Make sure counts for each level pair are high enough

$y = \beta_0 + x_1\beta_1 + x_2\beta_2 + x_3\beta_3 + x_4\beta_4 + x_5\beta|5 + x_6\beta_6$

\section{Analysis of Covariance (ANCOVA)}

Add ``covariates''

Adding ``continuous factors''

Add one more predictor (per covariate)

\[ X = \left( 
\begin{array}{ccccc}
C & 1 & 2 & 3 & age \\
1 & 0 & 0 & 1 & 23 \\
0 & 1 & 0 & 0 & 33 \end{array}
\right) \]


\end{document}





